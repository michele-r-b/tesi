\chapter{Evoluzione dei Protocolli di Comunicazione IoT}

\section{Introduzione ai protocolli IoT}
I protocolli di comunicazione IoT costituiscono il cuore tecnologico dei sistemi domotici intelligenti, definendo le modalità con cui dispositivi eterogenei si scambiano informazioni all'interno di una rete. L'evoluzione di tali protocolli riflette i cambiamenti delle esigenze applicative e tecnologiche, oltre che l'espansione delle applicazioni IoT nella vita quotidiana. Questi protocolli possono essere distinti principalmente in due categorie: protocolli cablati (come KNX, RS-485) e protocolli wireless (come Zigbee, Z-Wave, Wi-Fi, Bluetooth Low Energy, Thread e Matter).

\section{Protocolli cablati: KNX e RS-485}
Storicamente, le prime soluzioni domotiche si basavano prevalentemente su protocolli cablati, tra cui KNX e RS-485, che offrivano vantaggi in termini di stabilità e affidabilità della comunicazione. KNX, in particolare, è diventato uno standard internazionale (ISO/IEC 14543-3), noto per la sua robustezza, affidabilità e flessibilità, consentendo l'integrazione di diversi dispositivi all'interno di una rete domotica estesa \parencite{knxStandard}. RS-485, invece, ha trovato applicazione in contesti industriali e domestici grazie alla sua elevata resistenza alle interferenze e alla capacità di gestire connessioni su lunghe distanze.

\section{Protocolli wireless: Zigbee, Z-Wave, Wi-Fi e Bluetooth Low Energy}
Con l'avvento di tecnologie wireless più avanzate e la necessità di semplificare le installazioni domestiche, protocolli come Zigbee, Z-Wave, Wi-Fi e Bluetooth Low Energy (BLE) hanno acquisito crescente popolarità:
\begin{itemize}
    \item \textbf{Zigbee}: Basato sullo standard IEEE 802.15.4, è ottimizzato per reti mesh a basso consumo energetico, ideale per sensori e attuatori a batteria \parencite{zigbeeAlliance}.
    \item \textbf{Z-Wave}: Proprietario, ma molto diffuso nelle applicazioni domotiche residenziali grazie alla sua semplicità di configurazione, sicurezza integrata e bassi consumi energetici \parencite{zwaveAlliance}.
    \item \textbf{Wi-Fi}: Basato sullo standard IEEE 802.11, ha il vantaggio di una elevata velocità di trasmissione dati e ampia diffusione nelle abitazioni, ma soffre di elevato consumo energetico rispetto ad altri protocolli \parencite{wifiAlliance}.
    \item \textbf{Bluetooth Low Energy (BLE)}: Caratterizzato da consumi energetici estremamente ridotti, BLE è adatto per applicazioni a corto raggio, come dispositivi indossabili e sensori di prossimità \parencite{bluetoothSIG}.
\end{itemize}

\section{Thread e Matter: verso l'interoperabilità e l'unificazione}
Thread e Matter rappresentano l'evoluzione più recente nel panorama dei protocolli IoT domestici:
\begin{itemize}
    \item \textbf{Thread}: Protocollo wireless basato su IPv6, pensato specificatamente per IoT domestico, combina robustezza, bassa latenza e bassi consumi energetici con la flessibilità della rete mesh \parencite{threadGroup}.
    \item \textbf{Matter}: Sviluppato dalla Connectivity Standards Alliance (CSA), nasce dall'esigenza di creare un ecosistema interoperabile tra dispositivi multimarca utilizzando protocolli basati su IP (Internet Protocol). Matter promette di risolvere le barriere di interoperabilità, rendendo più semplice per i consumatori l’integrazione di nuovi dispositivi nei propri ecosistemi esistenti \parencite{matterAlliance}.
\end{itemize}

\section{Criteri di selezione dei protocolli}
La scelta del protocollo ideale dipende da vari fattori, tra cui:
\begin{itemize}
    \item Consumo energetico: particolarmente importante nei dispositivi alimentati a batteria.
    \item Portata e copertura: determinano la capacità del protocollo di garantire una comunicazione stabile in diversi contesti abitativi.
    \item Velocità e latenza: cruciali per applicazioni che richiedono risposte rapide e precise.
    \item Facilità d'uso e integrazione: influenzano direttamente l'esperienza utente e la scalabilità del sistema.
    \item Sicurezza e privacy: essenziali per proteggere le reti domestiche da intrusioni e garantire la riservatezza dei dati.
\end{itemize}

Nel prossimo capitolo saranno approfonditi ulteriormente gli aspetti cruciali della sicurezza e della privacy associati ai protocolli IoT domestici, esaminando anche le best practice per garantire reti domestiche sicure e affidabili.