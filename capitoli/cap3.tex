\chapter{Evoluzione dei Protocolli di Comunicazione IoT}
\section{Introduzione ai protocolli IoT}
I protocolli di comunicazione IoT sono insiemi di regole che permettono ai dispositivi intelligenti di scambiarsi dati all'interno di una rete. Nella domotica residenziale, essi rappresentano il fondamento tecnico per l'interoperabilità e l'efficienza dell'intero sistema. Tali protocolli si dividono principalmente in due categorie: cablati (come KNX, RS-485) e wireless (come Zigbee, Z-Wave, Wi-Fi, Bluetooth).

\section{Storia ed evoluzione dei protocolli}
Inizialmente, i sistemi domotici si basavano su protocolli proprietari, sviluppati da singoli produttori e caratterizzati da limitata interoperabilità. Con il tempo, la richiesta di soluzioni più flessibili ha favorito la nascita di standard aperti e interoperabili. L'evoluzione tecnologica ha portato all'adozione di reti mesh, integrazione con il cloud e, più recentemente, all'impiego di protocolli basati su IP, come Thread e Matter.

\section{Analisi dei principali protocolli}
\begin{itemize}
    \item \textbf{Zigbee}: protocollo wireless a basso consumo, basato su IEEE 802.15.4, adatto per reti mesh.
    \item \textbf{Z-Wave}: simile a Zigbee ma con maggiore interoperabilità, sviluppato specificamente per la domotica.
    \item \textbf{Wi-Fi}: ampiamente diffuso, offre elevata larghezza di banda ma consuma più energia.
    \item \textbf{Thread}: protocollo IPv6-based, progettato per reti mesh sicure e scalabili.
    \item \textbf{Matter}: standard aperto e universale, focalizzato sull'interoperabilità tra dispositivi multimarca.
    \item \textbf{Bluetooth Low Energy (BLE)}: adatto per comunicazioni a corto raggio e basso consumo.
\end{itemize}

\section{Innovazioni recenti}
Le innovazioni nel campo delle comunicazioni includono l'introduzione del 5G, che offre maggiore velocità e bassa latenza, e l'uso delle reti mesh per una copertura più affidabile. Inoltre, l'intelligenza artificiale viene sempre più integrata nei protocolli per ottimizzare la gestione della rete, migliorare la sicurezza e abilitare automazioni intelligenti basate sul comportamento dell'utente.

% Capitolo 4: Gestione di Dispositivi Multimarca