\chapter{Prospettive Future nella Domotica Residenziale}

\section{Introduzione}

Dopo aver esaminato nel Capitolo 5 le prestazioni, l'affidabilità e la scalabilità dei principali protocolli di comunicazione impiegati nelle abitazioni intelligenti, ora andiamo ad esplorare le tecologie emergenti per le smart home.

Negli ultimi anni, la casa ha smesso di essere solo il nostro spazio abitato, ma si sta trasformando progressivamente in un ecosistema interattivo. Questo cambiamento segue una trasformazione culturale e tecnologica più ampia, che rispecchia ilnostro modo di vivere, oramai siamo abituati a lavoriamo e interagiamo con l’ambiente che ci circonda. La domotica quindinon è più appannaggio di pochi appassionati o early adopter ma è diventata una componente concreta dell’abitare contemporaneo.

Questa evoluzione tuttavia non è lineare né priva di ostacoli. L’interoperabilità tra dispositivi, la protezione dei dati personali, l’efficienza energetica e l’inclusività sono sfide reali. Ma sono anche i punti su cui costruire una nuova idea di casa.\\

Dopo anni in cui i vari produttori fornivano soluzioni frammentate e proprietarie, oggi una nuova visione orientata sta guidando gli sviluppi concentrandosi su interoperabilità, sostenibilità e intelligenza diffusa. Questa transizione non riguarda soltanto l’arrivo di nuovi dispositivi e nuove tecnologie, ma è un cambiamento più profondo, la casa interagisce con i suoi abitanti impara da loro e si adatta in modo intelligente.

Le abitazioni del futuro saranno ambienti realmente smart, capaci di prevedere i nostri bisogni, reagire in tempo reale, proteggere dati sensibili e garantire accessibilità a tutti. In questo capitolo analizziamo le nuove tendenze emergenti, valutandone potenzialità e sfide.

\section{Lo standard Matter e i protocolli IP-native}

\subsection{Matter: interoperabilità come fondamento}

Matter non è soltanto un nuovo standard, ma è il simbolo di una storica alleanza tra le principali case produttrici dell’industria tecnologica quali Apple, Google, Amazon, Samsung, per superare i limiti della frammentazione e della interoperatività. Nasce come evoluzione del progetto CHIP (Connected Home over IP), con l'obiettivo di rendere compatibili tra loro dispositivi di diversi produttori.

Il punto di forza di Matter è basato su un modello applicativo standardizzato, che rende possibile un dialogo strutturato e uniforme tra tutti i componenti della rete domestica. La configurazione di nuovi dispositivi avviene in modo semplice e tramite dei wizard attivabili tramite QR code o NFC, supporta inolte il controllo multi-ecosistema, questo rende possibile l’utilizzo dello stesso dispositivo con più assistenti vocali o app differenti, senza conflitti, ad esmpio in casa posso controllare una luce sia da Siri che da Alexa. Inoltre, Matter supporta e privilegia il controllo locale, riuscendo così a garantire continuità anche in assenza di connessione Internet.

Basato sul protocollo di rete IPv6 e protetto dallo standard di crittografia TLS 1.3, Matter combina la scalabilità con la sicurezza.

\subsection{Thread: architettura distribuita e resiliente}

Il protocollo Thread, spesso utilizzato in abbinamento con Matter, è un protocollo basato su rete mesh IP-native, progettato specificatamente per l'utilizzo in ambiente IoT. Tra le sue caratteristiche principali, ogni nodo ha un indirizzo IP univoco, la rete si auto-configura e si auto-ripara, senza necessità di un coordinatore centrale. In caso di guasto del'Hub centrale leader, un nuovo leader viene eletto automaticamente, garantendo affidabilità e continuità.

Thread, come già analizzato per la sua architettura mesh e sicurezza nativa, si afferma ad oggi come la componente centrale per le reti domestiche, in grado di garantire resilienza ed estendibilità, soprattutto in abbinamento allo standard Matter 

L’integrazione di Thread in dispositivi come Apple HomePod, Alexa e Google Home ne facilitano l’adozione domestica, senza bisogno di doversi munire di Hub dedicati.

\section{Tecnologie emergenti per la casa intelligente}

\subsection{Intelligenza artificiale nella vita domestica}

L'adozione dell’intelligenza artificiale sta trasformando l’esperienza abitativa da una serie di automatismi e scenari che apprendono, anticipano e si adattano al nostro stile di vita.

\textbf{Esempio evolutivo}: oggi noi quando rientriamo in casa, diciamo "Alexa, accendi la luce"; un domani non tanto lontano, la casa riconoscerà il nostro rientro, rileverà la scarsa luminosità e accenderà automaticamente la luce più adatta, predisponendo l’ambiente secondo le nostre abitudini.

L’adozione del \textit{machine learning on-device} permette l’elaborazione dei dati direttamente sul dispositivo, riducendo la dipendenza dal cloud e la trasmissione di dati. Questo porta benefici concreti come una minore latenza, una maggiore privacy ed una personalizzazione maggiore.

\subsubsection{Apprendimento federato}

Attraverso il \textit{federated learning}, i dispositivi intelligenti elaboreranno i dati localmente e comunicheranno solo gli aggiornamenti del modello, così garantiranno che nessuna informazione sensibile verrà mai trasferita o memorizzata nel cloud.

\subsection{Reti mesh e Wi-Fi di nuova generazione}

Dopo aver evidenziato i benefici di una rete mesh nel capitolo precedente, le reti mesh oggi evolvono con l’integrazione dei nuovi standard Wi-Fi e Thread, abilitando così scenari più reattivi e modulari.


\textbf{Wi-Fi 6, 6E e 7} rappresentano tappe fondamentali nell’evoluzione delle reti:

\begin{itemize}
    \item \textbf{Wi-Fi 6}: introduce OFDMA, TWT e BSS Coloring, aumentando efficienza e riducendo la latenza.
    \item \textbf{Wi-Fi 6E}: aggiunge la banda a 6 GHz, liberando nuovi canali per dispositivi smart.
    \item \textbf{Wi-Fi 7}: promette latenze ultra-basse e velocità fino a 46 Gbps.
\end{itemize}

Una breve tabella riassuntiva può essere utile nel capitolo:

\begin{table}[H]
\centering
\begin{tabular}{lccc}
\toprule
\textbf{Versione} & \textbf{Anno} & \textbf{Banda} & \textbf{Caratteristiche principali} \\
\midrule
Wi-Fi 5 & 2014 & 2.4 / 5 GHz & Alta velocità, no ottimizzazioni IoT \\
Wi-Fi 6 & 2019 & 2.4 / 5 GHz & OFDMA, TWT, efficienza migliorata \\
Wi-Fi 6E & 2020 & 6 GHz & Nuovi canali, meno interferenze \\
Wi-Fi 7 & 2024 & 2.4 / 5 / 6 GHz & MLO, latenza bassa, throughput massimo \\
\bottomrule
\end{tabular}
\caption{Evoluzione dello standard Wi-Fi per ambienti smart home}
\end{table}

\subsection{Edge e fog computing domestico}

L’elaborazione locale dei dati tramite fog computing consente:
\begin{itemize}
    \item Risposte in tempo reale
    \item Resilienza in caso di assenza di Internet
    \item Protezione dei dati sensibili
    \item Riduzione del traffico e costi cloud
\end{itemize}

Progetti come K3s, una distribuzione leggera e semplificata di Kubernetes, pensata per ambienti IoT permetteranno l’orchestrazione di microservizi direttamente sugli hub domestici, trasformando così ogni stanza in un nodo intelligente.

\section{Privacy, sicurezza e fiducia digitale}

\subsection{Le nuove sfide dell’abitazione intelligente}

Sebbene l’intelligenza artificiale migliori l’automazione e l’efficienza dele nostre case, essa richiede un accesso costante ai nostri dati sensibili come le nostre abitudini quotidiane, i pattern di presenza nella casa e le nostre preferenze personali. Questa dipendenza  impone l’adozione di diverse rigorose misure di sicurezza e un approccio responsabile al trattamento dei dati.

\subsection{Strategie di protezione}

\begin{itemize}
    \item \textbf{Zero Trust Architecture}: ogni richiesta è validata, anche all’interno della rete.
    \item \textbf{Segmentazione delle reti}: per limitare i danni in caso di compromissione.
    \item \textbf{Crittografia omomorfica}: elabora dati cifrati senza decifrarli.
    \item \textbf{Blockchain per audit trail}: tracciabilità sicura delle azioni eseguite.
\end{itemize}
\section{Inclusività come principi guida}

\subsection{Tecnologia accessibile a tutti}

Domotica inclusiva significa:
\begin{itemize}
    \item Interfacce adattive per anziani e disabili
    \item Supporto vocale, gestuale e aptico
    \item Soluzioni retrofit per abitazioni esistenti
    \item Promozione dell’open source e del fai-da-te
\end{itemize}

\section{Conclusioni: la casa come alleata}

Non si tratta più solo di rendere la casa “smart”, ma di saper costruire un ambiente che sia in grado di ascoltare, apprendere e rispondere in modo etico e personalizzato. Il futuro della domotica non si misura dal numero di dispositivi installati in una casa, ma dalla loro capacità di adattarsi all’ambiente domestico in cui sono inseriti, di rispettare e migliorare la vita dei suoi abitanti.

