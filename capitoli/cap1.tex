
\chapter{Introduzione}
Negli ultimi anni, le tecnologie legate all'Internet of Things (IoT) hanno cambiato in modo significativo la nostra vita quotidiana, soprattutto all'interno delle nostre abitazioni. Ormai, la casa non è più semplicemente un luogo fatto di pareti e mobili, ma è diventata uno spazio intelligente e connesso, capace di adattarsi in modo dinamico alle nostre esigenze grazie alla domotica. Un esempio semplice e concreto è la possibilità di accendere il riscaldamento a distanza prima di rientrare dal lavoro, rendendo così la casa confortevole al nostro arrivo e risparmiando energia quando siamo fuori.

\vspace{0.5cm}
La domotica non è una novità recente: le sue prime forme risalgono agli anni '80, quando si usavano sistemi cablati piuttosto rudimentali, come il protocollo X10, per accendere e spegnere luci o elettrodomestici a distanza. Negli anni successivi, specialmente negli anni '90 e nei primi anni 2000, la tecnologia ha fatto passi da gigante. Sono comparsi sistemi come KNX e reti wireless come Zigbee e Z-Wave, che hanno reso possibile un controllo più semplice e integrato della casa. Probabilmente, molti ricordano il lancio dei termostati intelligenti come Nest o delle lampadine smart Philips Hue, prodotti che hanno reso tangibili i vantaggi della domotica per moltissime famiglie.

\vspace{0.5cm}
Recentemente, grazie all’intelligenza artificiale (IA), la domotica ha raggiunto un livello ancora più evoluto. Oggi non parliamo più solo di controllare dispositivi tramite app, ma di case capaci di imparare dalle nostre abitudini e adattarsi automaticamente ai nostri comportamenti quotidiani. Gli assistenti vocali come Amazon Alexa o Google Assistant, per esempio, non solo rispondono ai comandi, ma riescono a prevedere i nostri bisogni, suggerendoci persino routine personalizzate in base al giorno della settimana o al meteo.

\vspace{0.5cm}
Un'altra innovazione importante è l’edge computing, che permette ai dispositivi di elaborare i dati direttamente all’interno della rete domestica, senza passare per il cloud. Questo rende il sistema molto più veloce e affidabile. Ad esempio, una telecamera di sicurezza dotata di edge computing può riconoscere istantaneamente una persona sconosciuta e inviare subito una notifica, senza aspettare la risposta di un server remoto.

\vspace{0.5cm}
Le reti di nuova generazione come il 5G, e in futuro il 6G, aprono ulteriori opportunità, rendendo possibili applicazioni finora impensabili. Immaginiamo una casa dove possiamo usare realtà aumentata per controllare dispositivi o monitorare la nostra salute grazie a sensori che comunicano con il medico in tempo reale. Questi sviluppi promettono di migliorare ulteriormente il comfort abitativo e l’efficienza energetica, anche se pongono nuove sfide in termini di sicurezza e protezione della privacy.

\vspace{0.5cm}
Nonostante questi grandi vantaggi, una delle difficoltà più sentite dagli utenti rimane l’interoperabilità tra dispositivi di marche diverse. Chiunque abbia provato a installare nuovi dispositivi smart probabilmente ha sperimentato la frustrazione di dover gestire app e protocolli diversi. Fortunatamente, negli ultimi anni si sono diffusi standard aperti come il protocollo Matter, che mirano proprio a facilitare l’integrazione di dispositivi di marche differenti in un unico ecosistema.

\vspace{0.5cm}
Questa tesi si concentra proprio sull'evoluzione dei protocolli di comunicazione IoT nella domotica residenziale, con particolare attenzione a sicurezza, prestazioni e interoperabilità. Per rendere l'analisi più pratica e vicina alla realtà quotidiana, presenterò anche un caso concreto basato su Apple HomeKit, scelto per la sua semplicità d'uso e il suo buon livello di sicurezza.

% Nuova sezione metodologica più personale e discorsiva
\section{Panoramica metodologica della ricerca}
Il mio approccio alla ricerca è multidisciplinare: ho combinato l’analisi teorica dei protocolli IoT con la valutazione comparativa di dati provenienti da studi esistenti e test di laboratorio. In particolare, la metodologia seguita prevede:

\begin{itemize}
\item Una revisione approfondita della letteratura scientifica e tecnica disponibile;
\item Una comparazione dei protocolli più diffusi basata su dati concreti e test pratici;
\item L'analisi di casi reali e soluzioni commerciali, con particolare attenzione al sistema Apple HomeKit;
\item La realizzazione di un prototipo di sistema domotico multimarca, per verificare personalmente le problematiche e le potenzialità individuate durante lo studio teorico.
\end{itemize}

Questa metodologia mi ha permesso di offrire una panoramica completa e aggiornata della domotica residenziale, con spunti pratici utili a chiunque voglia avvicinarsi con consapevolezza al mondo della casa intelligente.

\section{Obiettivi della ricerca}
Gli obiettivi principali del mio lavoro sono:
\begin{itemize}
\item Ripercorrere in maniera chiara e completa la storia e l’evoluzione tecnologica dei protocolli IoT nella domotica;
\item Esaminare criticamente i problemi di sicurezza e privacy dei sistemi IoT domestici, proponendo soluzioni pratiche;
\item Valutare e confrontare le prestazioni dei protocolli più utilizzati tramite indicatori concreti come latenza, consumo energetico e scalabilità;
\item Studiare e descrivere strategie efficaci per garantire l'interoperabilità tra dispositivi di marche diverse;
\item Fornire un esempio pratico con Apple HomeKit, che dimostri nella realtà i concetti analizzati teoricamente.
\end{itemize}
