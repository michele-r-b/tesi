\chapter{Introduzione}
Negli ultimi anni, le tecnologie legate all’Internet of Things, o IoT, hanno fatto davvero tanta strada. Oggi si trovano praticamente ovunque e hanno cambiato il modo in cui viviamo la casa. Non si parla più solo di pareti e mobili: l’abitazione diventa un ambiente intelligente, connesso, capace di adattarsi ai bisogni quotidiani di chi la abita. La domotica, in questo senso, è forse l’esempio più evidente e concreto dell’IoT applicato alla vita di tutti i giorni: ci aiuta a risparmiare energia, ci fa sentire più sicuri e rende l’esperienza domestica più comoda.
\vspace{0.5cm}
Naturalmente, non è tutto perfetto. Man mano che la tecnologia è diventata più presente nelle case, sono emersi anche alcuni problemi. Uno tra i più sentiti riguarda la comunicazione tra dispositivi di marche diverse. Capita spesso, ad esempio, di acquistare un nuovo sensore o un assistente vocale e accorgersi che non “parla” con gli altri dispositivi già installati. Il motivo? Ogni produttore tende a usare protocolli di comunicazione propri, spesso incompatibili tra loro. E qui entra in gioco l’importanza dei cosiddetti standard aperti. Uno su tutti: Matter, un protocollo recente nato proprio per risolvere questi problemi e semplificare l’interoperabilità tra dispositivi.

\vspace{0.5cm}
Questa tesi nasce con l’idea di fare un po’ di chiarezza su tutto questo. L’obiettivo è analizzare in modo approfondito l’evoluzione dei protocolli di comunicazione utilizzati nella domotica residenziale, soffermandosi soprattutto su come si possano gestire insieme, in modo efficace, dispositivi di marche diverse. Per rendere tutto più concreto, sarà presentato anche un esempio pratico: un sistema domotico basato su Apple HomeKit, che mostra come creare una configurazione moderna, scalabile e, soprattutto, interoperabile.

\section{Obiettivi della ricerca}
Il cuore di questo lavoro è capire come funzionano i protocolli di comunicazione usati nei sistemi domotici e quale ruolo abbiano nel garantire che dispositivi diversi riescano a collaborare tra loro. Più nello specifico, ci si propone di:
\begin{itemize}
	 \item  Ripercorrere l’evoluzione dei principali protocolli IoT impiegati nell’ambito domestico, con uno sguardo alla loro origine e ai contesti in cui si sono affermati;
	 \item Confrontare le soluzioni attualmente disponibili, evidenziando i punti di forza e le criticità di ciascuna;
	 \item Capire quali sono le strade percorribili per mettere insieme dispositivi di marche diverse in un unico ecosistema funzionante;
	 \item Mostrare un caso concreto tramite la configurazione di un sistema domotico basato su Apple HomeKit, per osservare come funziona nella pratica l’integrazione di più tecnologie.
\end{itemize}

In sintesi, lo scopo di questo lavoro è offrire una panoramica il più possibile chiara e aggiornata sullo stato della domotica residenziale oggi. Si cercherà di dare risposte pratiche e di aprire una riflessione su come costruire sistemi domestici davvero intelligenti, in cui dispositivi diversi possano finalmente parlare la stessa lingua.

% Capitolo 2: La Domotica Residenziale