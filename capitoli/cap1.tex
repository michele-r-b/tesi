
\chapter{Introduzione}
Negli ultimi anni, le tecnologie legate all'Internet of Things (IoT) hanno trasformato profondamente il nostro modo di vivere, con un impatto particolarmente significativo sulla dimensione domestica. Oggi la casa non è più soltanto uno spazio fisico composto da mura e oggetti statici, ma un ambiente intelligente, connesso, capace di adattarsi dinamicamente alle esigenze degli abitanti grazie alla domotica. Quest'ultima rappresenta uno degli esempi più concreti e tangibili dell'applicazione dell'IoT nella vita quotidiana, migliorando il comfort abitativo, incrementando la sicurezza e favorendo il risparmio energetico.

\vspace{0.5cm}
Nonostante questi notevoli vantaggi, l'espansione della domotica ha anche evidenziato criticità non trascurabili. Tra queste, la sfida più rilevante è quella dell'interoperabilità tra dispositivi prodotti da aziende differenti. È frequente, infatti, acquistare nuovi dispositivi intelligenti e scoprire che non riescono a comunicare efficacemente con quelli già presenti in casa. La radice di questo problema risiede principalmente nella varietà di protocolli di comunicazione proprietari adottati dai diversi produttori, che limitano l'integrazione e la scalabilità delle soluzioni domotiche \parencite{matterAlliance}.

\vspace{0.5cm}
In risposta a questa situazione, negli ultimi anni si sono affermati nuovi standard aperti, come il protocollo Matter, sviluppato con l'obiettivo esplicito di facilitare l'interoperabilità tra dispositivi multimarca, basandosi su tecnologie IP (Internet Protocol) comuni \parencite{csaMatter}. Tuttavia, oltre all'interoperabilità, esistono altre dimensioni fondamentali come la sicurezza, la privacy e l'affidabilità, che determinano la reale efficacia e sostenibilità delle soluzioni IoT domestiche.

Questa tesi si propone quindi di esplorare approfonditamente l'evoluzione dei protocolli di comunicazione IoT adottati nell'ambito della domotica residenziale, affrontando in modo particolare i temi della sicurezza, delle prestazioni e della gestione efficace di dispositivi multimarca. Al fine di rendere l'analisi ancora più concreta e operativa, sarà presentato un caso pratico di sistema domotico basato sull'ecosistema Apple HomeKit, selezionato per la sua capacità di garantire un buon equilibrio tra facilità di uso, interoperabilità e sicurezza.

\section{Obiettivi della ricerca}
Questo lavoro ha come obiettivo principale analizzare in modo approfondito i protocolli IoT utilizzati nella domotica residenziale, evidenziando come la loro evoluzione stia influenzando la capacità dei sistemi domestici intelligenti di gestire in modo integrato dispositivi diversi. In particolare, gli obiettivi specifici di questa ricerca sono:
\begin{itemize}
\item Ripercorrere l'evoluzione storica e tecnologica dei principali protocolli di comunicazione IoT, analizzandone origine, diffusione e contesti applicativi;
\item Esaminare criticamente la sicurezza e la privacy dei sistemi IoT domestici, identificando minacce comuni e soluzioni efficaci;
\item Valutare le prestazioni e l'affidabilità dei protocolli più diffusi, confrontandoli attraverso indicatori chiave come latenza, consumo energetico e scalabilità;
\item Analizzare strategie concrete di gestione dell'interoperabilità, mostrando come creare ecosistemi domestici funzionali e scalabili;
\item Presentare un esempio applicativo basato sull'ecosistema Apple HomeKit, per offrire un riscontro pratico dei concetti teorici trattati.
\end{itemize}

In sintesi, questa tesi intende fornire una panoramica completa e aggiornata sullo stato attuale della domotica residenziale, offrendo spunti concreti per una gestione efficace e consapevole dei dispositivi intelligenti, in un contesto in cui la tecnologia IoT diventa sempre più centrale nella vita quotidiana.
