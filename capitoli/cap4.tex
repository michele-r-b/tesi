
\chapter{Sicurezza e Privacy nella Domotica Residenziale}

\section{Introduzione alla sicurezza IoT domestica}
La sicurezza in ambito IoT residenziale è cruciale poiché i dispositivi intelligenti raccolgono e condividono continuamente dati sensibili degli utenti, come informazioni personali, abitudini domestiche e dati biometrici. La crescente diffusione di dispositivi connessi, quali termostati intelligenti, telecamere di sorveglianza, assistenti vocali e sensori ambientali, aumenta la superficie di attacco potenziale. È quindi fondamentale adottare strategie di sicurezza integrate che includano la protezione dei dati in transito e a riposo, la gestione degli accessi e la resilienza contro attacchi informatici \parencite{roman2013, sicari2015}.

\section{Minacce e vulnerabilità comuni}
Le minacce tipiche nel contesto IoT domestico includono intrusioni non autorizzate, attacchi malware specifici per dispositivi embedded, e vulnerabilità nei protocolli di comunicazione wireless come ZigBee, Z-Wave, Wi-Fi e Bluetooth. Ad esempio, attacchi di tipo Man-in-the-Middle (MitM) possono intercettare e manipolare dati sensibili, mentre exploit di buffer overflow possono compromettere i firmware dei dispositivi. Inoltre, la presenza di dispositivi con password di default o aggiornamenti mancanti facilita l'accesso illecito. Un caso noto è stato l'attacco Mirai, che ha sfruttato dispositivi IoT vulnerabili per generare un vasto botnet DDoS \parencite{miraiBotnet}.

\section{Best practice per garantire sicurezza e privacy}
Per garantire sicurezza e privacy è consigliabile implementare una serie di misure preventive e protettive. Tra queste, l'applicazione regolare di aggiornamenti firmware e patch di sicurezza è fondamentale per correggere vulnerabilità note. La segmentazione della rete domestica, ad esempio tramite VLAN o reti Wi-Fi separate per dispositivi IoT e dispositivi personali, riduce il rischio di compromissione trasversale. L'uso di firewall e sistemi di rilevamento delle intrusioni (IDS) consente di monitorare il traffico e bloccare attività sospette. Inoltre, l'adozione di politiche di gestione degli accessi basate su principi di minimo privilegio e autenticazione multifattoriale (MFA) rafforza la protezione degli account. Infine, la sensibilizzazione degli utenti su pratiche di sicurezza, come la modifica delle password di default, è un elemento chiave \parencite{sicari2015, yang2017}.

\section{Tecniche di crittografia e autenticazione nei protocolli IoT}
La protezione dei dati in ambienti IoT si basa su tecniche di crittografia end-to-end, che garantiscono la riservatezza e l'integrità delle comunicazioni tra dispositivi e server cloud. Il protocollo TLS/SSL è ampiamente utilizzato per cifrare le comunicazioni su IP, mentre protocolli specifici per IoT, come DTLS, sono adottati per ambienti con limitate risorse computazionali. L'autenticazione a due fattori (2FA) e l'uso di certificati digitali migliorano la sicurezza degli accessi, riducendo il rischio di compromissione da password deboli. La gestione sicura delle credenziali, tramite hardware security modules (HSM) o Trusted Platform Modules (TPM), è essenziale per prevenire furti di chiavi crittografiche. Inoltre, l'adozione di standard come OAuth 2.0 e MQTT con autenticazione integrata facilita un controllo granulare degli accessi \parencite{sicari2015, yang2017}.

\section{Analisi di casi di violazione della sicurezza in ambito domestico}
Diversi casi concreti evidenziano le conseguenze di violazioni della sicurezza in ambito domestico. Un esempio è l'hacking di telecamere IP domestiche che ha portato alla diffusione di video privati online, sfruttando vulnerabilità nelle credenziali di default e firmware obsoleti \parencite{ringIncident}. Un altro caso riguarda la compromissione di assistenti vocali che, una volta controllati da attaccanti, possono intercettare conversazioni o attivare dispositivi senza consenso \parencite{nestIncident}. Tali studi contribuiscono a definire linee guida per la progettazione di sistemi più sicuri e resilienti.
