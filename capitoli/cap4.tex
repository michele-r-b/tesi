\chapter{Sicurezza e Privacy nella Domotica Residenziale}

\section{Introduzione alla sicurezza IoT domestica}

La tecnologia ha reso le nostre case più comode e facili da gestire: dal riscaldamento controllato a distanza alle luci che si regolano da sole. Ma insieme a questi benefici ci sono anche aspetti meno visibili, come la grande quantità di dati personali che questi sistemi raccolgono e trattano ogni giorno.

È interessante riflettere sulla quantità di informazioni che fluiscono attraverso una casa intelligente: orari di presenza, preferenze climatiche, abitudini di illuminazione, fino ad arrivare ai dati biometrici raccolti dalle telecamere di ultima generazione. Ogni componente del sistema - dal termostato intelligente all'assistente vocale - rappresenta contemporaneamente un'opportunità e una potenziale vulnerabilità.\\

Ogni nuovo dispositivo connesso aggiunge un “punto d’ingresso” alla nostra rete domestica. Non dobbiamo più pensare alla sicurezza di un solo apparecchio, ma di un sistema dove tutto comunica con tutto, spesso anche con servizi online. Questa rete invisibile all’occhio dell’utente richiede strategie di protezione completamente riviste.\\

L'approccio più efficace prevede l'integrazione della sicurezza fin dalle fasi iniziali di progettazione - il cosiddetto principio del \textit{security by design}. Questo significa implementare protezioni a più livelli: cifratura dei dati in transito e a riposo, gestione granulare dei permessi, meccanismi di difesa adattivi capaci di rispondere a minacce in evoluzione.

\section{Minacce e vulnerabilità comuni}

Le minacce alla domotica hanno dinamiche tutte loro, diverse da quelle della sicurezza informatica “classica”. Capire queste differenze è il primo passo per proteggere davvero la propria casa smart.

\subsection{Intrusioni e accessi non autorizzati}

Un aspetto sorprendentemente critico riguarda la presenza di credenziali di default nei dispositivi IoT che non vengono aggiornate. Nonostante anni di sensibilizzazione, numerosi produttori continuano a distribuire dispositivi con combinazioni username/password facilmente reperibili attraverso una semplice ricerca online. Questa pratica, unita alla tendenza degli utenti a non modificare tali credenziali, crea vulnerabilità immediate e facilmente sfruttabili.

La situazione è aggravata dalla mancanza di meccanismi che obblighino l'utente a personalizzare le credenziali al primo utilizzo - una misura semplice che potrebbe eliminare gran parte di questi rischi.

\subsection{Malware specifici per dispositivi embedded}

I dispositivi IoT, caratterizzati da risorse computazionali limitate e sistemi operativi minimali, presentano un profilo di vulnerabilità unico. I malware progettati per questi ambienti sfruttano proprio queste limitazioni: la scarsa capacità di implementare antivirus tradizionali, l'impossibilità di monitorare in tempo reale i processi in esecuzione, la difficoltà nell'applicare patch di sicurezza.

Questi software malevoli possono operare inosservati per periodi prolungati, trasformando dispositivi apparentemente innocui in strumenti per la raccolta di dati sensibili o in nodi di botnet per attacchi distribuiti.

\subsection{Vulnerabilità nei protocolli di comunicazione}

L'eterogeneità dei protocolli wireless nella domotica - ZigBee, Z-Wave, Wi-Fi, Bluetooth - introduce sfide specifiche di sicurezza. Gli attacchi di tipo Man-in-the-Middle rappresentano una minaccia particolarmente insidiosa in questo contesto. Un attore malevolo può posizionarsi nel percorso di comunicazione tra dispositivi, intercettando e potenzialmente alterando i comandi trasmessi. Consideriamo l'esempio di una serratura intelligente: l'intercettazione dei segnali di controllo potrebbe permettere l'apertura della casa in un secondo momento.

\subsection{Il caso Mirai: una lezione da non dimenticare}

L'epidemia del botnet Mirai nel 2016 rimane un caso di studio fondamentale per comprendere le vulnerabilità sistemiche dell'IoT. Questo malware ha dimostrato come la combinazione di credenziali predefinite e mancanza di aggiornamenti di sicurezza possa trasformare centinaia di migliaia di dispositivi domestici in armi per attacchi DDoS di scala globale. La semplicità dell'attacco - basato essenzialmente sul tentativo sistematico di credenziali note - evidenzia come problemi apparentemente banali possano avere conseguenze devastanti \parencite{miraiBotnet}.

\section{Best practice per garantire sicurezza e privacy}

Per difendere un ambiente domotico serve una strategia a più livelli, che unisca soluzioni tecniche, buone pratiche e formazione degli utenti. Insieme, questi elementi creano un sistema capace di reagire e adattarsi ai rischi che cambiano nel tempo.

\subsection{Aggiornamenti e patch management}

La gestione sistematica degli aggiornamenti rappresenta la prima linea di difesa contro vulnerabilità note. Questo processo, apparentemente semplice, presenta sfide pratiche significative nell'ambito IoT: molti dispositivi non implementano meccanismi di aggiornamento automatico, richiedendo interventi manuali periodici. La creazione di una routine di verifica - magari calendarizzata mensilmente - può trasformare questa attività da sporadica emergenza a pratica consolidata.

\subsection{Segmentazione della rete}

L'isolamento logico dei dispositivi attraverso la segmentazione di rete offre benefici significativi con un impegno implementativo relativamente contenuto:

\begin{itemize}
    \item \textbf{Rete principale protetta}: riservata a dispositivi contenenti dati sensibili e sistemi IoT verificati
    \item \textbf{Rete ospiti isolata}: per visitatori occasionali, impedendo accessi non autorizzati all'infrastruttura principale
\end{itemize}

Questa separazione limita la propagazione di eventuali compromissioni e facilita il monitoraggio del traffico anomalo.

\subsection{Firewall e sistemi di monitoraggio}

I router consumer moderni hanno fatto passi significativi nell'integrazione di funzionalità di sicurezza precedentemente riservate ad ambienti enterprise. Produttori come ASUS, Netgear e AVM (Fritz!Box) offrono oggi:
    \
\begin{itemize}
    \item Sistemi di prevenzione delle intrusioni (IPS) integrati
    \item Analisi comportamentale del traffico di rete
    \item Filtri per contenuti malevoli aggiornati in tempo reale
    \item Sistemi di notifica per eventi di sicurezza rilevanti
\end{itemize}

L'attivazione di queste funzionalità, spesso disponibili ma disabilitate di default, può incrementare significativamente il livello di protezione con uno sforzo minimo. Alcuni provider hanno iniziato a fornire dispositivi preconfigurati con queste protezioni attive, semplificando ulteriormente l'adozione.

\subsection{Gestione degli accessi e autenticazione forte}

L'implementazione di politiche di accesso robuste richiede un bilanciamento tra sicurezza e usabilità:

\begin{itemize}
    \item \textbf{Principio del privilegio minimo}: assegnare solo i permessi strettamente necessari per ogni utente o dispositivo
    \item \textbf{Autenticazione multi-fattore (MFA)}: integrare fattori di autenticazione aggiuntivi, bilanciando sicurezza e praticità d'uso
    \item \textbf{Gestione centralizzata delle credenziali}: l'utilizzo di password manager facilita l'adozione di credenziali complesse e uniche
    \item \textbf{Rotazione programmata}: stabilire intervalli regolari per l'aggiornamento delle credenziali critiche
\end{itemize}

\subsection{Educazione e consapevolezza degli utenti}

La componente umana rimane fondamentale in qualsiasi strategia di sicurezza. La formazione degli abitanti della casa, soprattutto per gli utenti con ruoli di gestione amministrativa dei dispositivi,  dovrebbe coprire:

\begin{itemize}
    \item Riconoscimento di tentativi di phishing specifici per dispositivi IoT
    \item Comprensione dell'importanza degli aggiornamenti di sicurezza
    \item Capacità di verificare l'autenticità di app e servizi collegati
    \item Identificazione di comportamenti anomali nei dispositivi
\end{itemize}

Sessioni informative periodiche, magari integrate con esempi pratici e simulazioni, possono trasformare ogni membro della famiglia in un elemento attivo del sistema di sicurezza.

\section{Tecniche di crittografia e autenticazione nei protocolli IoT}

L'implementazione di meccanismi crittografici in ambienti con risorse limitate rappresenta una delle sfide tecniche più interessanti della sicurezza IoT.

\subsection{Crittografia end-to-end nei sistemi domotici}

La protezione crittografica dei dati deve essere garantita lungo l'intero percorso di trasmissione, dal dispositivo Ioy fino al nostro smartphone. Le limitazioni hardware dei dispositivi IoT impongono scelte oculate nell'implementazione:

\begin{itemize}
    \item \textbf{AES-128}: uno degli standard più diffusi, usato perché offre un buon equilibrio tra sicurezza e velocità. Protegge bene senza pesare troppo sulle prestazioni o sulla batteria.
    \item \textbf{Crittografia a curve ellittiche (ECC)}: garantisce lo stesso livello di protezione di altri sistemi ma con chiavi più corte, quindi più veloce e adatta anche ai dispositivi più piccoli.
    \item \textbf{Suite crittografiche leggere}: algoritmi come ChaCha20-Poly1305, pensati apposta per l’IoT, che mantengono alta la sicurezza anche su hardware con risorse limitate.
\end{itemize}
\subsection{Protocolli di comunicazione sicura}

L'adattamento dei protocolli di sicurezza tradizionali alle necessità e caratteristiche dei dispositivi IoT ha prodotto soluzioni innovative:

\begin{itemize}
    \item \textbf{DTLS (Datagram TLS)}: una versione di TLS pensata per funzionare bene con il protocollo UDP, utile per dispositivi che si connettono in modo intermittente o con reti non stabili.
    \item \textbf{Protocolli sicuri a livello applicativo}: come CoAP-DTLS e MQTT-TLS, che integrano le funzioni di sicurezza direttamente nel protocollo usato dai dispositivi per comunicare.
    \item \textbf{Meccanismi di attestazione}: sistemi che controllano l’integrità e l’autenticità del dispositivo prima di consentire lo scambio di dati, così da evitare comunicazioni con unità compromesse.
\end{itemize}

\subsection{Standard di autenticazione nell'era IoT}

La convergenza verso standard aperti facilita l'interoperabilità dei dispositivi di produttori differenti senza compromettere la sicurezza:

\begin{itemize}
    \item \textbf{OAuth 2.0 e OpenID Connect}: permettono di concedere permessi specifici a un servizio senza dover condividere direttamente la password, aumentando la sicurezza.
    \item \textbf{JWT e CBOR Web Tokens}: piccoli “pacchetti” di informazioni firmati digitalmente che contengono tutto ciò che serve per l’autenticazione, riducendo la necessità di mantenere dati lato server.
    \item \textbf{FIDO2/WebAuthn}: sistemi di autenticazione senza password, che usano metodi come impronte digitali o riconoscimento facciale, sempre più adottati anche nei dispositivi IoT.
\end{itemize}

La selezione di dispositivi che implementano questi standard non solo garantisce maggiore sicurezza ma anche migliore integrazione futura.

\section{Analisi di casi di violazione della sicurezza in ambito domestico}

L'esame di incidenti reali fornisce informazioni preziose per la prevenzione di future compromissioni.

\subsection{Il caso delle telecamere IP compromesse}

L'incidente del 2020 che ha esposto migliaia di feed video domestici rappresenta un caso emblematico e su vasta scala:

\begin{itemize}
    \item Persistenza di credenziali di accesso di default 
    \item Nessun aggiornamento del firmware da diversi anni
    \item Esposizione diretta delle porte di gestione dei servizi critici direttamente su Internet
    \item Assenza di cifratura per lo stream dei video
\end{itemize}

L'analisi successiva ha rivelato come la concatenazione di vulnerabilità apparentemente minori possa creare brecce di sicurezza maggiori.

\subsection{L'incidente Ring e le implicazioni sulla privacy}

Il caso Ring del 2019 ha evidenziato come la sicurezza debba estendersi oltre il dispositivo stesso. Gli hacker hanno sfruttato diversi punti di vulnerabiltà:

\begin{itemize}
    \item Riutilizzo di credenziali compromesse in precedenti data breach per lo stesso account personale
    \item Mancata adozione di MFA nonostante fosse disponibile
    \item Scarsa attenzione degli utenti agli indicatori di compromissione
\end{itemize}

La risposta di Amazon - rendere obbligatoria l'autenticazione a due fattori dopo l'azione legale collettiva - dimostra come la pressione regolatoria e sociale possa accelerare l'adozione di misure di sicurezza basilari ma efficaci.

\subsection{Lezioni apprese e raccomandazioni}

L'analisi trasversale di questi incidenti rivela pattern ricorrenti:

\begin{enumerate}
    \item \textbf{Security by default}: la configurazione sicura deve essere lo stato iniziale, non un'opzione
    \item \textbf{Trasparenza proattiva}: gli utenti devono comprendere quali dati vengono raccolti e come vengono protetti
    \item \textbf{Modello di responsabilità condivisa}: successo richiede collaborazione tra produttori, provider e utenti finali
    \item \textbf{Resilienza operativa}: piani di incident response testati minimizzano l'impatto di eventuali breach
\end{enumerate}

\section{Conclusioni e prospettive future}

In ambito domotico, la sicurezza va vista come un processo in continua evoluzione. Ogni progresso tecnologico porta con sé nuove opportunità, ma anche rischi che richiedono un aggiornamento costante delle strategie di difesa.

Le direzioni future più promettenti includono:

\begin{itemize}
    \item \textbf{Intelligenza artificiale per la sicurezza adattiva}: sistemi che apprendono pattern comportamentali per identificare anomalie in tempo reale
    \item \textbf{Tecnologie distributed ledger}: blockchain e simili per garantire integrità e non ripudiabilità dei dati IoT
    \item \textbf{Crittografia post-quantistica}: preparazione proattiva all'era del quantum computing
\end{itemize}

Vogliamo abitazioni intelligenti che offrano tutti i vantaggi della tecnologia, ma senza sacrificare la sicurezza e la riservatezza dei dati e l'usabilità dei dispositivi. E' un traguardo possibile grazie all'adozione di pratiche consolidate, all’uso di consapevole di soluzioni innovative per la sicurezza e alla formazione continua di chi le utilizza.