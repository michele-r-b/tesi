\chapter{Gestione di Dispositivi Multimarca}

\section{Introduzione}

Immagina di essere in un negozio di elettronica. Vuoi rendere la tua casa più smart e finalmente ti decidi: una bella lampadina connessa, magari Philips. Poi noti un termostato intelligente, quello della Nest ti convince. La serratura smart più sicura? Yale. E già che ci sei, aggiungi anche un paio di telecamere Arlo, che sembrano avere ottime recensioni.

Torni a casa soddisfatto, pronto a configurare tutto. Ma in pochi minuti la sensazione cambia: ogni dispositivo richiede la sua app, il suo account, il suo hub. Le lampadine non parlano con il termostato, la serratura ignora le telecamere, e tu inizi a sentire che più che in una casa intelligente, ti trovi in una casa... \textbf{complicata}.

Ecco il paradosso della domotica moderna: più scelta abbiamo, più difficile diventa mettere tutto insieme. Questo capitolo entra nel cuore di questa sfida. Perché far dialogare dispositivi di marche diverse non è solo un problema tecnico: significa garantire coerenza nell’esperienza d’uso, mantenere sicurezza e affidabilità, e soprattutto non perdere la testa ogni volta che si aggiunge qualcosa di nuovo.
\section{La sfida dell'interoperabilità}

\subsection{Le radici del problema}
Quando pensiamo all’interoperabilità nella domotica, potremmo immaginarla come un semplice problema tecnico da risolvere. Ma la realtà è molto più complessa. Questa frammentazione nasce da anni — anzi, decenni — di scelte industriali fatte non tanto per facilitare la vita degli utenti, quanto per rafforzare la posizione di mercato dei produttori. Ogni azienda ha costruito il proprio ecosistema come un ``giardino recintato'', dove tutto funziona bene finché resti dentro, ma diventa complicato appena provi a integrare qualcosa di diverso.


\subsubsection{La Torre di Babele dei protocolli}
La situazione attuale assomiglia a una moderna Torre di Babele: ogni produttore parla la propria lingua, rendendo difficile – se non impossibile – far comunicare dispositivi diversi tra loro.

\begin{itemize}
\item \textbf{Protocolli proprietari}: Ogni grande azienda ha sviluppato il proprio "dialetto". Lutron usa ClearConnect, Insteon ha un protocollo dual-band tutto suo, Somfy parla RTS o io-homecontrol. Il risultato? Dispositivi eccellenti, ma incapaci di capirsi tra loro.

\item \textbf{Varianti su uno stesso tema}: Anche quando si sceglie lo stesso linguaggio di base, come Zigbee, non è detto che ci sia comprensione. Philips Hue adotta Zigbee Light Link, altri preferiscono Zigbee Home Automation: stessi fondamentali, ma grammatica e accento differenti, che spesso non si comprendono.

\item \textbf{Livelli diversi di astrazione}: Alcuni protocolli lavorano "sotto il cofano", a livello fisico (come Z-Wave), altri agiscono più in superficie, a livello di esperienza utente e applicazioni (come HomeKit). Tradurre da uno all’altro è come passare da codice macchina a conversazione naturale: servono interpreti intelligenti.

\item \textbf{Sicurezza che non si parla}: Anche i modelli di sicurezza variano. Diverse tecniche di crittografia e autenticazione rendono difficile, e talvolta rischioso, mettere in comunicazione dispositivi di produttori diversi senza compromettere la protezione dei dati.
\end{itemize}

\subsubsection{L'impatto sull'utente finale}

La frammentazione dell’ecosistema domotico si traduce, per l’utente finale, in un’esperienza spesso frustrante e tutt’altro che “smart”:

\textbf{Troppe app, poca chiarezza}: Secondo una ricerca del 2023, l’utente medio di una casa intelligente utilizza tra le 8 e le 12 app diverse per controllare i propri dispositivi. Ogni produttore ha la sua, ognuna con interfaccia diversa, logiche diverse, notifiche diverse. Il risultato? Il telefono diventa un campo di battaglia digitale, e avere una visione d’insieme del sistema è praticamente impossibile.

\textbf{Automazioni che faticano a cooperare}: Una semplice regola, come “quando esco di casa, spegni tutte le luci e abbassa il termostato”, può trasformarsi in un puzzle complicato. Ogni pezzo della catena – lampadine, riscaldamento, sensori di presenza – parla una lingua diversa, e spesso non si capiscono. Una parte funziona, l’altra no. E l’utente si ritrova a dover intervenire manualmente.

\textbf{Lentezza imprevista}: Ogni volta che un comando deve attraversare più livelli – ad esempio da un hub Zigbee, a un bridge proprietario, poi al cloud del produttore, da lì a un altro servizio cloud, e infine al dispositivo – si accumulano ritardi. Quello che dovrebbe essere un’azione immediata può richiedere diversi secondi. Non è solo fastidioso, ma anche poco affidabile.

\textbf{Costi invisibili, ma reali}: Oltre al prezzo dei dispositivi, spesso bisogna acquistare hub aggiuntivi, bridge, gateway, e magari anche sottoscrivere abbonamenti per funzionalità cloud avanzate. Una casa “veramente smart” può arrivare a richiedere 3, 4 o 5 hub diversi per funzionare come si desidera.

\subsection{L'evoluzione verso standard comuni}

Per anni, la domotica ha viaggiato su binari paralleli, con ogni produttore convinto di poter costruire un ecosistema chiuso e autosufficiente. Tuttavia, col tempo è diventato evidente che questa frammentazione non giova a nessuno: complica la vita agli utenti, rallenta l’adozione di massa e genera una percezione di tecnologia “instabile” o poco matura. Quando persino i tecnici iniziano a faticare per far dialogare tra loro i dispositivi, è chiaro che serve un cambio di paradigma.

Per fortuna, una parte sempre più ampia del settore ha riconosciuto che l’interoperabilità non è solo un vantaggio competitivo, ma una necessità. E così è iniziata un’evoluzione importante, guidata da due forze complementari: da un lato la spinta delle comunità open source, dall’altro la convergenza delle grandi aziende su standard comuni.

\subsubsection{Il movimento open source}

Nel vuoto lasciato da un’industria ancora troppo frammentata, la community open source ha saputo costruire soluzioni concrete, flessibili e sorprendentemente avanzate. Sviluppatori indipendenti, appassionati, maker e professionisti IT si sono uniti per creare piattaforme capaci di integrare decine – se non centinaia – di dispositivi diversi, superando le barriere imposte dai vendor.

Ecco alcune delle iniziative più significative:

\begin{itemize}
    \item \textbf{Home Assistant}: Nato nel 2013 come progetto personale di Paulus Schoutsen, è diventato oggi il punto di riferimento per l’integrazione multimarca. Con oltre 2000 integrazioni supportate, permette di unificare luci, termostati, sensori, elettrodomestici e molto altro in un’unica interfaccia \cite{HomeAssistant2024}.

    \item \textbf{OpenHAB}: Un altro gigante del mondo open source, pensato fin dall’inizio con un’architettura modulare orientata alla flessibilità. Permette di scrivere logiche complesse in linguaggi come JavaScript, Groovy o Python, rendendolo ideale per chi ha esigenze articolate \cite{OpenHAB2023}.

    \item \textbf{Node-RED}: Una piattaforma di programmazione visuale che ha conquistato il cuore di molti utenti tecnici. Permette di creare automazioni collegando nodi con un semplice drag-and-drop, visualizzando in tempo reale il flusso dei dati. È spesso utilizzata in combinazione con Home Assistant per estendere ulteriormente le capacità \cite{NodeRED2024}.
\end{itemize}

Queste soluzioni hanno dimostrato che l’interoperabilità è possibile già oggi, senza attendere che tutti i produttori si accordino su uno standard. E, soprattutto, hanno dato voce e potere agli utenti, rendendoli protagonisti attivi nella costruzione del proprio ecosistema domotico.


\subsubsection{L'alleanza dell'industria: Matter}

Nel mondo dell’elettronica di consumo, vedere Apple, Google, Amazon e Samsung seduti allo stesso tavolo non è solo raro: è quasi impensabile. Eppure è successo. Dopo anni di ecosistemi chiusi, incompatibilità frustranti e guerre silenziose per il controllo della casa connessa, le grandi aziende hanno riconosciuto che la vera innovazione non passa (più) dall’esclusione, ma dalla collaborazione.

Da questo storico compromesso è nato \textbf{Matter}, uno standard aperto che ambisce a diventare il “linguaggio comune” della domotica. Non si tratta semplicemente di un nuovo protocollo, ma di un vero cambio di paradigma: finalmente i dispositivi possono parlarsi senza traduttori, senza ponti complicati, e senza la necessità di scegliere da subito "da che parte stare".

Il percorso per arrivare a Matter non è stato breve né semplice:

\begin{itemize}
    \item \textbf{2019}: Nasce \textit{Project CHIP} (Connected Home over IP), inizialmente come iniziativa tecnica congiunta per risolvere i problemi di compatibilità più evidenti.
    \item \textbf{2021}: Il progetto cambia nome in \textit{Matter} e vengono pubblicate le prime specifiche ufficiali.
    \item \textbf{2022}: I primi dispositivi certificati fanno la loro comparsa sul mercato, seguiti da aggiornamenti firmware che permettono anche a dispositivi esistenti di essere compatibili.
    \item \textbf{2023-2024}: L'adozione cresce rapidamente, con il supporto esteso in tutti i principali ecosistemi e una base di dispositivi sempre più ampia.
\end{itemize}

Ciò che rende Matter davvero rivoluzionario è l'approccio mentalmente inclusivo. Invece di obbligare l’utente a scegliere tra HomeKit, Google Home o Alexa, Matter consente che uno stesso dispositivo venga controllato da più piattaforme contemporaneamente. Una lampadina smart può essere aggiunta sia all’iPhone che all’altoparlante Google, e rispondere a entrambi. Non è più l’utente a doversi adattare al sistema, ma il contrario.

Ma c’è anche un risvolto strategico: il successo di Matter non è solo un favore fatto agli utenti, ma anche un riconoscimento, da parte dei big tech, che la casa connessa non può più essere una somma di silos isolati. Serve coerenza, semplicità, trasparenza.

Matter segna, quindi, un passaggio da un’epoca di ``lock-in'' tecnologici a un’era di collaborazione e apertura. Una scelta coraggiosa, ma necessaria, per fare della domotica non un lusso per esperti, ma una realtà accessibile a tutti \cite{ConnectivityStandardsAlliance2023}.


\subsubsection{Architettura di un gateway universale}

Un gateway universale è, in sostanza, il poliglotta della casa intelligente. Il suo compito è ascoltare e tradurre, in tempo reale, i messaggi che i dispositivi si scambiano in lingue diverse: Zigbee, Z-Wave, Bluetooth, Wi-Fi, Thread. Ma per farlo bene, deve essere costruito con intelligenza e precisione.

Sotto la sua scocca minimalista si nasconde una piccola centrale operativa, composta da:

\begin{enumerate}
    \item \textbf{Radio multiple}: Il cuore della comunicazione. Ogni protocollo ha le sue frequenze e peculiarità. Un gateway moderno integra chip radio separati per Zigbee, Z-Wave, Bluetooth, a volte anche infrarossi o 433/868 MHz, con antenne calibrate per garantire la massima stabilità del segnale.
    
    \item \textbf{Processore potente}: Tradurre al volo i comandi tra decine di dispositivi richiede una buona dose di potenza di calcolo. La maggior parte dei gateway si affida a processori ARM (es. Cortex-A53 o A72) capaci di gestire automazioni, crittografia e comunicazioni cloud/locali senza rallentamenti.
    
    \item \textbf{Stack software modulare}: Qui risiede la vera intelligenza del sistema. Ogni protocollo ha il suo driver, ma tutti convergono verso un layer di astrazione che uniforma i comandi. Questo permette, ad esempio, di accendere una lampadina Z-Wave con un sensore Zigbee, senza che l’utente si debba preoccupare delle differenze sottostanti.
    
    \item \textbf{Storage locale}: Una piccola memoria interna tiene traccia dello stato dei dispositivi, delle preferenze utente, delle automazioni configurate. Questo garantisce che, anche in caso di blackout o disconnessione Internet, la casa continui a funzionare in modo coerente.
    
    \item \textbf{Connettività di rete}: Per dialogare con smartphone, assistenti vocali e servizi esterni, il gateway deve essere ben connesso. Ethernet è spesso preferita per stabilità, ma il Wi-Fi offre maggiore flessibilità. Alcuni modelli includono anche 4G di backup.
\end{enumerate}

L’obiettivo non è solo far funzionare tutto, ma farlo con discrezione. Il miglior gateway è quello che l’utente dimentica di avere, perché tutto “funziona e basta”.


\subsection{Piattaforme open-source: il potere della community}

Le piattaforme open-source hanno rivoluzionato la gestione multimarca, offrendo flessibilità e controllo senza precedenti.

\subsubsection{Home Assistant: il gigante dell'integrazione}

Home Assistant merita un'analisi approfondita per il suo impatto sull'ecosistema:

\textbf{Architettura componibile}:
\begin{itemize}
    \item \textbf{Core}: Scritto in Python, gestisce stato e eventi
    \item \textbf{Integrazioni}: Moduli per ogni marca/protocollo
    \item \textbf{Frontend}: Interfaccia web moderna con Lovelace UI
    \item \textbf{Add-ons}: Servizi aggiuntivi containerizzati
\end{itemize}

\textbf{Deployment flessibile}:
\begin{itemize}
    \item Home Assistant OS: Sistema operativo dedicato per Raspberry Pi
    \item Container Docker: Per integrazione in sistemi esistenti
    \item Supervised: Gestione add-on su Linux generico
    \item Core: Installazione Python pura per massimo controllo
\end{itemize}

\subsubsection{OpenHAB: l'alternativa enterprise}

OpenHAB si distingue per:

\begin{itemize}
    \item \textbf{Architettura OSGi}: Modulare e robusta, adatta a deployment mission-critical
    \item \textbf{Rules engine potente}: Supporta JavaScript, Groovy, Python per logiche complesse
    \item \textbf{HABPanel}: Interfaccia personalizzabile per tablet e display fissi
    \item \textbf{Bindings}: Oltre 400 integrazioni mantenute dalla community
\end{itemize}

\subsubsection{Node-RED: programmazione visuale per tutti}

Node-RED democratizza l'automazione complessa:

\begin{itemize}
    \item \textbf{Flow-based programming}: Trascinare nodi e collegarli con fili
    \item \textbf{Vasta libreria di nodi}: Per ogni protocollo e servizio
    \item \textbf{Debug visuale}: Vedere in tempo reale i dati che fluiscono
    \item \textbf{Estensibilità}: Creare nodi custom in JavaScript
\end{itemize}

\subsection{Lo standard Matter: unificazione nativa}

Matter non è solo un altro protocollo, ma un cambio di paradigma nella gestione multimarca.

\subsubsection{Caratteristiche rivoluzionarie}

\textbf{Multi-admin nativo}: Un dispositivo Matter può essere controllato simultaneamente da:
\begin{itemize}
    \item Apple HomeKit
    \item Google Home
    \item Amazon Alexa
    \item Samsung SmartThings
    \item Qualsiasi altro controller Matter
\end{itemize}

Questo elimina la necessità di ``scegliere un campo'' al momento dell'acquisto.

\textbf{Commissioning unificato}: Setup tramite:
\begin{itemize}
    \item QR code standard su ogni dispositivo
    \item NFC tap per dispositivi compatibili
    \item Codice numerico come fallback
\end{itemize}

Il processo è identico indipendentemente dalla piattaforma usata.

\textbf{Tipi di dispositivi standardizzati}: Matter definisce ``device types'' comuni:
\begin{itemize}
    \item On/Off Light
    \item Dimmable Light
    \item Color Temperature Light
    \item Thermostat
    \item Door Lock
    \item Window Covering
    \item E molti altri...
\end{itemize}

Ogni tipo ha attributi e comandi standard, garantendo comportamento uniforme.

\section{Soluzioni native basate sugli smartphone}

Gli ecosistemi mobile hanno evoluto le loro piattaforme domotiche da semplici app di controllo a veri e propri sistemi operativi per la casa.

\subsection{Apple HomeKit: la fortezza della privacy}

Apple ha costruito HomeKit con la privacy come principio fondamentale, differenziandosi nettamente dalla concorrenza.

\subsubsection{Architettura security-first}

\textbf{Crittografia end-to-end}: Ogni comunicazione tra iPhone e dispositivo è cifrata con chiavi uniche per sessione. Neanche Apple può decifrare i comandi.

\textbf{Processing locale}: Le automazioni girano su HomePod, Apple TV o iPad designato come hub. Nessun cloud coinvolto per operazioni base.

\textbf{HomeKit Secure Video}: Video delle telecamere analizzati localmente per riconoscere persone, animali, veicoli. Solo notifiche cifrate vanno su iCloud.

\subsubsection{Esperienza utente raffinata}

\textbf{App Casa}: Interface minimalista con:
\begin{itemize}
    \item Vista per stanze con preview live
    \item Controlli adattivi (slider per luci, termostato circolare)
    \item Scene predefinite modificabili
    \item Automazioni con logica condizionale
\end{itemize}

\textbf{Siri integration}: Comandi naturali come:
\begin{itemize}
    \item ``Ehi Siri, buonanotte'' → spegne luci, abbassa temperatura, attiva allarme
    \item ``Sto arrivando a casa'' → accende riscaldamento basandosi su posizione
    \item ``Com'è la situazione a casa?'' → riassunto stato dispositivi
\end{itemize}

\subsubsection{Limitazioni e workaround}

La rigidità di HomeKit ha pro e contro:

\textbf{Limitazioni}:
\begin{itemize}
    \item Solo dispositivi certificati MFi (costosi)
    \item Numero limitato di automazioni condizionali
    \item Nessun accesso web (solo app iOS/macOS)
\end{itemize}

\textbf{Workaround community}:
\begin{itemize}
    \item Homebridge: Software che simula un bridge HomeKit, permettendo integrazione dispositivi non certificati
    \item Shortcuts app: Automazioni avanzate che triggerano scene HomeKit
    \item Home+ app: Client alternativo con funzioni avanzate
\end{itemize}

\cite{Apple2023}

\subsection{Google Home: l'intelligenza del machine learning}

Google applica la sua expertise in AI e ML per creare un ecosistema predittivo e proattivo.

\subsubsection{Funzionalità AI-powered}

\textbf{Home/Away Assist}: Utilizza:
\begin{itemize}
    \item Posizione di tutti i telefoni familiari
    \item Pattern di movimento da sensori
    \item Calendario condiviso
    \item Storico comportamentale
\end{itemize}

Per determinare automaticamente quando attivare modalità ``casa vuota''.

\textbf{Routine suggerite}: L'AI analizza comportamenti e suggerisce:
\begin{itemize}
    \item ``Ho notato che accendi sempre queste luci insieme, vuoi creare una routine?''
    \item ``Il termostato è spesso troppo alto alle 22, vuoi che lo abbassi automaticamente?''
    \item ``Parti sempre alle 8:15, vuoi che prepari il caffè alle 8?''
\end{itemize}

\subsubsection{Integrazione ecosistema Google}

\textbf{Nest integration}: Dispositivi Nest hanno funzioni esclusive:
\begin{itemize}
    \item Nest Thermostat apprende preferenze e ottimizza consumi
    \item Nest Cam riconosce volti familiari vs estranei
    \item Nest Protect comunica con termostato per sicurezza
\end{itemize}

\textbf{Assistant everywhere}: Controllo vocale da:
\begin{itemize}
    \item Telefoni Android/iOS
    \item Smart speaker/display
    \item Android Auto
    \item Wear OS
    \item Android TV
\end{itemize}

\subsection{Amazon Alexa: l'ecosistema più vasto}

Amazon ha costruito l'ecosistema più ampio grazie a una strategia di apertura e prezzi aggressivi.

\subsubsection{Skills: app per la casa}

Il modello delle Skill permette espansione infinita:

\textbf{Skill ufficiali}: Ogni produttore può creare skill per:
\begin{itemize}
    \item Controllo vocale dispositivi
    \item Notifiche e alert
    \item Routine personalizzate
    \item Integrazioni con servizi
\end{itemize}

\textbf{Skill community}: Migliaia di skill create da:
\begin{itemize}
    \item Sviluppatori indipendenti
    \item Hobbisti
    \item Aziende di servizi
\end{itemize}

Esempio: Skill ``Casa Intelligente'' che aggiunge comandi italiani naturali.

\subsubsection{Hardware ecosystem}

Amazon produce dispositivi per ogni esigenza:

\begin{itemize}
    \item \textbf{Echo}: Speaker di ogni dimensione e prezzo
    \item \textbf{Echo Show}: Display per controllo visuale
    \item \textbf{Echo Hub}: Controller dedicato con dashboard
    \item \textbf{Ring}: Sicurezza integrata
    \item \textbf{Eero}: Mesh Wi-Fi con Zigbee integrato
\end{itemize}

\subsection{Samsung SmartThings: il veterano rinnovato}

SmartThings, acquisito da Samsung nel 2014, combina legacy e innovazione.

\subsubsection{Punti di forza unici}

\textbf{Edge computing}: SmartThings Edge sposta automazioni su hub locale:
\begin{itemize}
    \item Driver scaricabili per dispositivi
    \item Automazioni girano offline
    \item Latenza minimizzata
    \item Privacy migliorata
\end{itemize}

\textbf{Integrazione elettrodomestici}: Unico con controllo nativo di:
\begin{itemize}
    \item Lavatrici/asciugatrici smart
    \item Frigoriferi Family Hub
    \item Aspirapolvere robot
    \item TV e soundbar
\end{itemize}

\textbf{Ecosystem Galaxy}: Integrazione profonda con:
\begin{itemize}
    \item Smartphone Galaxy
    \item Galaxy Watch per presenza
    \item Galaxy Buds per audio spaziale
    \item Tablet come controller fissi
\end{itemize}

\cite{SmartThings}

\section{Il futuro della gestione multimarca}

\subsection{Trend emergenti}

\subsubsection{AI per l'interoperabilità}

L'intelligenza artificiale promette di rivoluzionare la gestione multimarca:

\textbf{Traduzione semantica}: AI che comprende l'intento dell'utente e lo traduce nei comandi specifici per ogni dispositivo, indipendentemente dal protocollo.

\textbf{Apprendimento delle incompatibilità}: Sistemi che imparano automaticamente workaround per far funzionare insieme dispositivi teoricamente incompatibili.

\textbf{Ottimizzazione automatica}: AI che analizza l'uso e suggerisce migrazioni verso piattaforme più adatte o configurazioni più efficienti.

\subsubsection{Standardizzazione post-Matter}

Mentre Matter risolve molti problemi, nuove sfide emergono:

\textbf{Matter 2.0 e oltre}:
\begin{itemize}
    \item Supporto per categorie dispositivi avanzate (robot, elettrodomestici maggiori)
    \item Gestione energia integrata per sostenibilità
    \item Interoperabilità con sistemi building automation
    \item Standard per AI e ML edge
\end{itemize}

\textbf{Convergenza con altri domini}:
\begin{itemize}
    \item Automotive (casa che prepara partenza)
    \item Salute (dispositivi medicali integrati)
    \item Energia (V2H, solar, battery storage)
    \item Smart city (servizi municipali integrati)
\end{itemize}

\subsection{Raccomandazioni per il futuro}

\subsubsection{Per i consumatori}

\begin{itemize}
    \item Preferire dispositivi con supporto Matter per futureproofing
    \item Investire in infrastruttura di rete robusta
    \item Mantenere sempre un livello di controllo manuale
    \item Documentare per continuità familiare
\end{itemize}

\subsubsection{Per l'industria}

\begin{itemize}
    \item Abbracciare standard aperti completamente
    \item Fornire migration path chiari da sistemi legacy
    \item Investire in UX per semplificare complessità
    \item Garantire supporto lungo termine (10+ anni)
\end{itemize}

\section{Conclusioni}

La gestione di dispositivi multimarca nella domotica residenziale rappresenta una delle sfide più concrete e al contempo stimolanti del settore. Dalla Torre di Babele iniziale dei protocolli proprietari, stiamo assistendo a una convergenza verso soluzioni più aperte e interoperabili.

Le piattaforme open source come Home Assistant hanno dimostrato che è possibile far dialogare dispositivi di qualsiasi marca, mentre standard come Matter promettono di rendere l'interoperabilità nativa e trasparente. Le soluzioni native degli smartphone, pur con i loro limiti, hanno reso la domotica accessibile a milioni di utenti non tecnici.

I casi studio presentati dimostrano che non esiste una soluzione unica: ogni implementazione deve essere calibrata sulle specifiche esigenze, competenze e vincoli. Che si tratti di un appartamento high-tech per un professionista, una casa famiglia con esigenze diverse, o un retrofit rispettoso di vincoli architettonici, la chiave sta nella pianificazione attenta e nell'implementazione graduale.

Il futuro della gestione multimarca sarà caratterizzato da maggiore intelligenza artificiale, standard più comprensivi e convergenza con altri domini tecnologici. Ma il principio fondamentale rimarrà invariato: la tecnologia deve adattarsi alle persone, non viceversa. Solo con questo approccio human-centric la promessa della casa intelligente potrà realizzarsi pienamente per tutti.