\chapter{La Domotica Residenziale}

\section{Definizione e principi fondamentali}
La domotica residenziale indica l'integrazione delle tecnologie elettroniche e informatiche per automatizzare, controllare e ottimizzare gli impianti e i dispositivi presenti nelle abitazioni. Questo campo applicativo sfrutta in maniera determinante l'Internet of Things (IoT), consentendo agli utenti un controllo sia locale che remoto degli ambienti domestici \parencite{domoticaWiki}. I principi fondamentali della domotica comprendono automazione, integrazione, personalizzazione e interoperabilità, aspetti che sono cruciali per il funzionamento efficace di un sistema intelligente.

\section{Vantaggi della domotica: efficienza energetica, sicurezza e comfort abitativo}
L'impiego della domotica in contesti residenziali porta numerosi vantaggi, tra cui spiccano l'efficienza energetica, l'aumento della sicurezza e il miglioramento del comfort abitativo. Sistemi intelligenti avanzati permettono, ad esempio, di ottimizzare automaticamente l'illuminazione e la climatizzazione sulla base di parametri ambientali e comportamenti abituali degli utenti, riducendo così significativamente i consumi energetici \parencite{iecSmartHome}. Sul fronte della sicurezza, sensori intelligenti e telecamere integrate consentono una sorveglianza continua, intervenendo autonomamente in caso di emergenze o situazioni anomale \parencite{nistIoTSecurity}. Il comfort è garantito da interfacce intuitive, quali app per dispositivi mobili e assistenti vocali, che rendono semplice e immediata la gestione personalizzata degli ambienti domestici.

\section{Componenti principali di un sistema domotico}
Un sistema domotico completo ed efficace è composto da diversi elementi essenziali che interagiscono costantemente tra loro \parencite{iecSmartHome}:
\begin{itemize}
    \item \textbf{Sensori intelligenti}: dispositivi in grado di rilevare parametri ambientali (temperatura, umidità, luminosità, movimento), fornendo dati essenziali per le automazioni;
    \item \textbf{Attuatori}: dispositivi che trasformano i comandi ricevuti in azioni concrete, come l'accensione o lo spegnimento di luci, regolazione di tapparelle o riscaldamento;
    \item \textbf{Unità centrale di controllo (hub o gateway)}: componente centrale del sistema che gestisce le regole di automazione, interpreta i dati dei sensori e coordina gli attuatori;
    \item \textbf{Interfacce utente}: comprendono applicazioni mobili, assistenti vocali o pannelli di controllo fisici, permettendo agli utenti di interagire facilmente con il sistema;
    \item \textbf{Rete di comunicazione}: infrastruttura di rete che collega i dispositivi domotici, solitamente basata su protocolli cablati (es. KNX) o wireless (es. Wi-Fi, Zigbee, Thread o Matter).
\end{itemize}

\section{Sfide aperte nella domotica residenziale}
Nonostante gli evidenti vantaggi, permangono diverse sfide cruciali per una diffusione più ampia e sostenibile della domotica, tra cui l'interoperabilità tra sistemi multimarca, la sicurezza informatica e l'affidabilità delle soluzioni implementate. Questi temi saranno approfonditi nei capitoli successivi, analizzando nello specifico l'importanza della sicurezza IoT e le prestazioni dei vari protocolli di comunicazione utilizzati nel contesto domestico.