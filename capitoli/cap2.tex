\chapter{La Domotica Residenziale}
\section{Definizione e principi fondamentali}
La domotica residenziale rappresenta l’applicazione delle tecnologie informatiche ed elettroniche alla gestione e automazione dell’abitazione. Essa consente di controllare, anche da remoto, impianti e dispositivi domestici, migliorando la qualità della vita degli utenti \parencite{domoticaWiki}. I principi fondamentali che regolano la domotica includono l’automazione, l’integrazione tra sistemi, la personalizzazione degli ambienti e l’interconnessione tra i dispositivi.

\section{Vantaggi della domotica: efficienza, sicurezza e comfort}
Tra i principali benefici della domotica si annoverano il miglioramento dell’efficienza energetica, l’incremento della sicurezza domestica e l’aumento del comfort abitativo. Sistemi avanzati permettono, ad esempio, la regolazione automatica dell’illuminazione e della climatizzazione sulla base di parametri ambientali e comportamenti abituali \parencite{iecSmartHome}. Sul piano della sicurezza, l’integrazione di sensori e telecamere consente un monitoraggio continuo e l’attivazione automatica di allarmi in caso di eventi anomali. Il comfort, infine, è garantito dalla possibilità di personalizzare ambienti e scenari mediante dispositivi mobili o assistenti vocali \parencite{domoticaWiki}.

\section{Componenti principali di un sistema domotico}
Un sistema domotico si compone di più elementi che interagiscono tra loro per garantire il funzionamento del sistema nel suo complesso \parencite{iecSmartHome}:
\begin{itemize}
    \item \textbf{Sensori}: rilevano parametri ambientali come temperatura, umidità, movimento, luminosità, ecc.;
    \item \textbf{Attuatori}: ricevono comandi e producono azioni fisiche (es. accensione luci);
    \item \textbf{Unità di controllo}: rappresentano il centro logico del sistema e gestiscono le regole di automazione;
    \item \textbf{Interfacce utente}: dispositivi o applicazioni che consentono il controllo da parte dell’utente;
    \item \textbf{Rete di comunicazione}: infrastruttura cablata o wireless per lo scambio di dati.
\end{itemize}


% Capitolo 3: Evoluzione dei Protocolli di Comunicazione IoT