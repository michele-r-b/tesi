\chapter{Caso di Studio: Sistema Domotico Integrato con Apple HomeKit}

\section{Obiettivi e contesto}
Questo caso di studio documenta l’implementazione di un sistema domotico per una residenza unifamiliare su due piani, integrato nativamente con Apple HomeKit. 

I requisiti riguardano: (i) controllo accessi sicuro; (ii) gestione dell’illuminazione e delle tapparelle; (iii) videosorveglianza perimetrale con notifiche mirate; (iv) monitoraggio ambientale e automazioni meteo; (v) audio multi stanza; (vi) facilità d’uso tramite app Casa e comandi vocali; (vii) riduzione dei consumi energetici senza impattare il comfort.

\section{Analisi dell’abitazione}
\textbf{Piano terra} (100 m²): ingresso, soggiorno open-space (40 m²), cucina abitabile (25 m²), bagno ospiti, studio/ufficio (15 m²), garage doppio.\\
\textbf{Piano primo} (100 m²): camera matrimoniale con bagno, due camere singole, bagno principale, terrazzo (20 m²).\\
\textbf{Esterno}: giardino perimetrale, vialetto d’accesso, due accessi carrabili, area barbecue.

Vincoli progettuali: distribuzione su due piani (copertura Wi‑Fi e latenza), esposizioni est/ovest (controllo solare), necessità di integrazione non invasiva su impianto esistente e preferenza per controllo nativo da ecosistema Apple. Facilità di gestione e condivisione con i membri della famiglia.

\section{Architettura del sistema}
L’architettura è organizzata su tre livelli: (a) \textit{dispositivi e sensori}; (b) \textit{hub/controller e gateway}; (c) \textit{automazioni e interfacce}.

\subsection{Rete e infrastruttura}
\begin{itemize}
  \item \textbf{LAN IoT dedicata}: VLAN 192.168.20.0/24, SSID dedicato (WPA3), segmentazione del traffico verso la LAN principale.
  \item \textbf{Hub HomeKit}: HomePod in soggiorno come hub primario; HomePod mini in cucina, studio e camera matrimoniale come hub di failover e per audio locale.
  \item \textbf{Gateway BTicino}: modulo DIN collegato alla rete Wi-Fi per la configurazione dei dispositivi su home control e condivisione con HomeKit.
\end{itemize}

\subsection{Protocolli e integrazione}
\begin{itemize}
  \item \textbf{Bus SCS (BTicino)}: utilizzato per la comunicazione tra moduli Living Now cablati (interruttori, dimmer e comandi tapparelle) e il gateway, quando l’impianto lo prevede. Garantisce bassa latenza e funzionalità anche in assenza di rete IP.
  \item \textbf{Wi-Fi}: impiegato dai dispositivi Living Now connessi \textit{pre-Matter} (installati in questa abitazione), dal Nuki Smart Lock 3.0 Pro, dalle telecamere Netatmo e dalla stazione meteo Netatmo. È il principale canale di comunicazione verso HomeKit in questa configurazione.
  \item \textbf{Thread}: viene utilizzato esclusivamente da dispositivi compatibili con lo standard Matter o nativamente basati su Thread, come ad esempio le \emph{Nanoleaf Lines}. La gestione della rete è affidata ai \textit{Border Router}, già integrati in dispositivi di largo consumo come l’\emph{HomePod mini} e l’\emph{HomePod} di seconda generazione. Questa architettura consente di ottenere comunicazioni a bassa latenza e con un consumo energetico ridotto, due caratteristiche fondamentali per garantire un’esperienza fluida e sostenibile all’interno della smart home.
  \item \textbf{Interruttori BTicino wireless}: modelli da parete privi di cablaggio, alimentati a batteria con comunicazione radio diretta con il gateway via Thread.
  \item \textbf{HomeKit su IP} rappresenta il protocollo applicativo di Apple dedicato all’automazione domestica e al controllo dei dispositivi smart. Si basa su un sistema di cifratura end-to-end, che assicura la protezione dei dati durante tutta la comunicazione, e utilizza un meccanismo di pairing semplice e sicuro tramite codice QR. Questo approccio riduce al minimo i rischi di accesso non autorizzato e rende l’esperienza di configurazione più intuitiva per l’utente finale.
  \item \textbf{HomeKit Secure Video (HKSV)}: per le telecamere Netatmo, con analisi video on-device e registrazione cifrata su iCloud.
\end{itemize}

\section{Dotazione installata (per categoria)}
\subsection*{Accesso e sicurezza}
\begin{itemize}
  \item \textbf{Serratura smart}: Nuki Smart Lock 3.0 Pro (integrazione HomeKit via Wi‑Fi). Funzioni: blocco/sblocco, stato porta, automazioni in base a presenza e orario. La gestione delle credenziali ospiti avviene nell’app Nuki; HomeKit fornisce controllo e notifiche.
  \item \textbf{Videosorveglianza esterna}: 4 telecamere Netatmo Outdoor con sirena. In HomeKit Secure Video: rilevamento persone/animali/veicoli, zone di attività, timeline cifrata. Posizionamento: ingresso, garage, giardino posteriore, lato secondario.
\end{itemize}

\subsection*{Illuminazione e schermature}
\begin{itemize}
  \item \textbf{Punti luce e dimmer}: serie BTicino Living Now per 32 punti luce (on/off e dimmerazione). Comandi scenari locali per richiamo rapido.
  \item \textbf{Tapparelle}: 18 motorizzazioni con controllo salita/discesa/stop e posizionamento percentuale; scenari alba/tramonto.
\end{itemize}

\subsection*{Monitoraggio ambientale e meteo}
\begin{itemize}
  \item \textbf{Stazione meteo Netatmo}: modulo interno (temperatura, umidità, CO\textsubscript{2}, pressione, rumore) e modulo esterno (temperatura/umidità). Dati usati per automazioni (protezione solare, avvisi vento/calore).
  \item \textbf{Sensori integrati HomePod}: temperatura/umidità ambientale disponibili in HomeKit per regole semplici di comfort.
\end{itemize}

\subsection*{Audio, interazione e notifiche}
\begin{itemize}
  \item \textbf{HomePod / HomePod mini}: audio multi-room, comandi vocali, Intercom, hub HomeKit con failover. Integrazione con scene (es. modalità cinema, sveglia graduale).
\end{itemize}

\subsection*{Illuminazione decorativa}
\begin{itemize}
  \item \textbf{Nanoleaf Lines}: installazione a parete in soggiorno (dietro TV). Integrazione con HomeKit per scene; effetti dinamici per notifiche e modalità cinema. Connessione di rete via Wi‑Fi; supporto a protocolli low-latency ove disponibile.
\end{itemize}

\section{Implementazione}
\subsection{Fase 1 — Infrastruttura elettrica e di rete}
\begin{enumerate}
  \item Installazione del gateway BTicino su barra DIN e predisposizione protezioni dedicate.
  \item Alimentazioni per telecamere esterne.
  \item Configurazione VLAN/SSID IoT, indirizzamento statico per dispositivi fissi (telecamere, gateway), WPA3.
\end{enumerate}

\subsection{Fase 2 — Installazione dispositivi}
\begin{itemize}
  \item Sostituzione interruttori tradizionali con moduli Living Now; configurazione dimmer nelle zone principali; comandi tapparelle.
  \item Montaggio Nuki su porta blindata con calibrazione.
  \item Installazione telecamere Netatmo e definizione delle zone di attività.
  \item Posizionamento HomePod/HomePod mini e installazione Nanoleaf Lines.
\end{itemize}

\subsection{Fase 3 — Configurazione HomeKit}
\begin{enumerate}
  \item Aggiunta del gateway BTicino tramite codice HomeKit; rilevamento automatico dei dispositivi collegati.
  \item Pairing di telecamere e stazione meteo Netatmo; abilitazione HKSV.
  \item Configurazione Nuki via QR e verifica controlli da app Casa.
  \item Impostazione HomePod come hub primario e verifica dei servizi remoti.
  \item Organizzazione in stanze/piani e definizione di zone (interno/esterno).
\end{enumerate}

\section{Automazioni rappresentative}
Le automazioni sono progettate con logica "\textit{evento-condizione-azione}", utilizzando trigger affidabili e preferendo esecuzione locale quando possibile.

\subsection*{Protezione notturna}
\begin{verbatim}
Trigger: tramonto + 30 min; presenza = true
Azioni: chiusura tapparelle piano terra; accensione luci esterne;
        telecamere in alta sensibilità; blocco automatico serratura.
Notifiche: sintesi eventi su iPhone dei genitori.
\end{verbatim}

\subsection*{Arrivo familiare (riconoscimento)}
\begin{verbatim}
Trigger: Netatmo ingresso -> persona riconosciuta; fascia oraria: 18:00-23:00
Azioni: luce ingresso 100% a 3000K; avviso su HomePod (Intercom);
        sblocco manuale facilitato (notifica azionabile da Apple Watch/iPhone).
\end{verbatim}

\subsection*{Ottimizzazione solare estiva}
\begin{verbatim}
Trigger: temperatura esterna > 26°C; meteo soleggiato
Condizioni: stagione = estate; qualcuno in casa = true
Azioni progressive: lato est 70% (10:00); lato sud 80% (12:00);
                   lato ovest 70% (15:00); apertura al tramonto.
\end{verbatim}

\subsection*{Sveglia graduale}
\begin{verbatim}
Trigger: 07:00 (feriali) / 08:30 (weekend)
Azioni: tapparelle camera 30% con apertura lenta; luce 20% a 2700K (5 min);
        HomePod: riproduzione playlist con volume in fade-in.
\end{verbatim}

\subsection*{Modalità cinema}
\begin{verbatim}
Trigger: comando vocale "Ehi Siri, modalità cinema" o pulsante scena
Azioni: chiusura tapparelle soggiorno; spegnimento luci;
        Nanoleaf Lines in Screen/Mirror a 30%;
        HomePod in modalità Home Theater; notifiche familiari silenziate (2 ore).
\end{verbatim}

\section{Gestione utenti, permessi e privacy}
\begin{itemize}
  \item \textbf{Ruoli HomeKit}: genitori come amministratori (gestione dispositivi, automazioni, registrazioni video); figli come membri con accesso a luci/tapparelle e audio nelle loro stanze; esclusi controlli su serratura e telecamere.
  \item \textbf{HKSV e dati}: Tra le funzionalità avanzate di HomeKit rientra l’analisi del movimento eseguita direttamente \textit{on-device}, che consente di elaborare i dati in locale senza doverli inviare a server esterni su cloud. I flussi video sono inoltre protetti grazie a una cifratura end-to-end e possono essere archiviati in modo sicuro su iCloud. L’utente ha anche la possibilità di definire \textbf{zone sensibili} all’interno dell’inquadratura, riducendo così il numero di notifiche superflue e concentrando l’attenzione solo sugli eventi realmente rilevanti.

  \item \textbf{Accessi ospiti}: gestione credenziali temporanee tramite app Nuki; revoca immediata; log accessi consultabile.
\end{itemize}

\section{Conclusioni: vantaggi dell'integrazione multimarca}
L'adozione di un'architettura HomeKit in un contesto multimarca, come quello descritto in questo caso di studio, ha permesso di ottenere un impianto coerente, sicuro e facile da gestire.

\paragraph{Principali vantaggi osservati}

Dall’analisi delle soluzioni integrate in ambiente HomeKit emergono diversi vantaggi rilevanti. Innanzitutto, la possibilità di una \textbf{gestione centralizzata} semplifica notevolmente l’esperienza d’uso: dispositivi eterogenei come quelli di BTicino, Netatmo, Nuki o Nanoleaf possono essere controllati da un’unica interfaccia, l’app Casa, evitando la frammentazione in applicazioni separate per ogni produttore.  

Un ulteriore beneficio riguarda la \textbf{condivisione semplificata}. Grazie alla funzione di Condivisione in famiglia di Apple, l’accesso ai dispositivi può essere esteso automaticamente a tutti i membri della famiglia, senza necessità di configurazioni aggiuntive.  

Particolarmente significativo è l’aspetto legato a \textbf{privacy e sicurezza}: HomeKit adotta nativamente meccanismi di cifratura end-to-end e automazioni locali. Nel caso dei flussi video, la funzione HomeKit Secure Video assicura che l’analisi venga eseguita direttamente on-device, con archiviazione protetta su iCloud.  

La piattaforma contribuisce inoltre alla \textbf{riduzione della complessità}, eliminando la necessità di gateway o bridge multipli, e centralizzando sia gli aggiornamenti sia le notifiche di sistema. A ciò si affianca una comprovata \textbf{affidabilità operativa}: luci, tapparelle e scenari principali continuano a funzionare anche in assenza di connessione Internet, garantendo continuità all’utente.  

L’\textbf{esperienza uniforme} rappresenta un ulteriore valore aggiunto, grazie a un’interfaccia coerente e comandi vocali disponibili su tutti i dispositivi Apple (iPhone, iPad, Mac e Apple Watch). Allo stesso tempo, l’ecosistema è orientato alla \textbf{scalabilità futura}: la presenza di border router integrati, come quelli degli HomePod, rende la rete pronta ad accogliere dispositivi Matter e Thread.  

Non mancano vantaggi in termini di \textbf{efficienza energetica}, con automazioni dedicate all’illuminazione e alle schermature solari che permettono di ridurre i consumi migliorando al tempo stesso il comfort abitativo. Infine, l’adozione di pratiche di \textbf{sicurezza di rete}, come la segmentazione VLAN per i dispositivi IoT e l’isolamento del traffico rispetto alla rete principale, contribuisce a rafforzare ulteriormente l’affidabilità dell’intero sistema.


\noindent In sintesi, la soluzione implementata dimostra come un ecosistema aperto e interoperabile, basato su standard consolidati come HomeKit, possa integrare tecnologie di produttori diversi garantendo facilità d'uso, sicurezza e possibilità di evoluzione nel tempo.