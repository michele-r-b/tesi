\chapter{Glossario dei termini e degli acronimi}
\begin{description}
    \item[Frammentazione] Situazione in cui dispositivi, piattaforme e protocolli di comunicazione non sono pienamente interoperabili a causa dell’adozione di standard, profili o implementazioni proprietarie differenti.
    \item[IoT] Insieme di dispositivi e sensori collegati a Internet, capaci di comunicare autonomamente e rendere intelligenti ambienti e oggetti quotidiani.
    \item[BLE] Bluetooth Low Energy --- Standard wireless a basso consumo energetico.
    \item[Zigbee] Protocollo wireless basato su IEEE 802.15.4, ottimizzato per reti mesh a corto raggio.
    \item[Z-Wave] Protocollo wireless a bassa potenza, usato per applicazioni di domotica.
    \item[Wi-Fi] Wireless Fidelity --- Tecnologia di rete locale senza fili basata su IEEE 802.11.
    \item[Thread] Protocollo di rete IPv6-based pensato per dispositivi IoT.
    \item[Matter] Standard aperto per l'interoperabilità tra dispositivi smart, sviluppato dalla CSA.
    \item[HomeKit] Piattaforma sviluppata da Apple per controllare e gestire in maniera semplice e sicura i dispositivi domestici intelligenti tramite dispositivi iOS.
    \item[Gateway] Dispositivo che consente la comunicazione tra reti o protocolli differenti.
    \item[API] Application Programming Interface --- Interfaccia che permette l'interazione tra software.
    \item[Hub] Dispositivo che agisce da centro di controllo, permettendo a diversi dispositivi smart di comunicare tra loro e con l'utente.
    \item[KNX] Standard aperto per l'automazione degli edifici, utilizzato principalmente in sistemi cablati per applicazioni domotiche.
    \item[Rete mesh] Tipologia di rete in cui ciascun dispositivo è collegato direttamente a più altri, aumentando l'affidabilità e l'efficienza nella comunicazione.
    \item[IPv6] Versione più recente del protocollo Internet, che consente un numero quasi illimitato di indirizzi IP.
    \item[Hub centrale] Dispositivo centrale che coordina e gestisce le comunicazioni tra dispositivi intelligenti in una rete domotica.
    \item[RS-485] Standard di comunicazione seriale cablato, resistente alle interferenze e utilizzato principalmente in ambienti industriali e domotici per connessioni su lunghe distanze.
    \item[VLAN] Rete virtuale che separa logicamente gruppi di dispositivi all'interno della stessa rete fisica, migliorando sicurezza e gestione.
    \item[IDS] Sistema che monitora il traffico di rete per identificare e segnalare eventuali intrusioni o attività sospette non autorizzate.
    \item[Firmware] Software fondamentale, installato permanentemente sui dispositivi hardware per controllare direttamente le loro operazioni di base.
    \item[DTLS] Datagram Transport Layer Security --- Protocollo di sicurezza che fornisce comunicazioni cifrate per dispositivi con risorse limitate, basato su UDP.
    \item[VLAN] Rete virtuale che separa logicamente gruppi di dispositivi all'interno della stessa rete fisica, migliorando sicurezza e gestione.
    \item[IDS] Sistema che monitora il traffico di rete per identificare e segnalare eventuali intrusioni o attività sospette non autorizzate.
    \item[Firmware] Software fondamentale, installato permanentemente sui dispositivi hardware per controllare direttamente le loro operazioni di base.
    \item[DTLS] Datagram Transport Layer Security --- Protocollo di sicurezza che fornisce comunicazioni cifrate per dispositivi con risorse limitate, basato su UDP.
    \item[Tag] Dispositivo utilizzato per attivare automazioni domotiche. Può essere passivo (NFC/RFID, senza batteria, funziona al tocco) o attivo (BLE/beacon, con batteria, rileva la presenza a distanza).
    \item[MFA] Multi-Factor Authentication – Metodo di autenticazione che richiede all’utente di fornire due o più prove (“fattori”) per verificare la propria identità prima di concedere l’accesso a un sistema o servizio
\end{description}